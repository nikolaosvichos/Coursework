% Options for packages loaded elsewhere
% Options for packages loaded elsewhere
\PassOptionsToPackage{unicode}{hyperref}
\PassOptionsToPackage{hyphens}{url}
%
\documentclass[
  ignorenonframetext,
]{beamer}
\newif\ifbibliography
\usepackage{pgfpages}
\setbeamertemplate{caption}[numbered]
\setbeamertemplate{caption label separator}{: }
\setbeamercolor{caption name}{fg=normal text.fg}
\beamertemplatenavigationsymbolsempty
% remove section numbering
\setbeamertemplate{part page}{
  \centering
  \begin{beamercolorbox}[sep=16pt,center]{part title}
    \usebeamerfont{part title}\insertpart\par
  \end{beamercolorbox}
}
\setbeamertemplate{section page}{
  \centering
  \begin{beamercolorbox}[sep=12pt,center]{section title}
    \usebeamerfont{section title}\insertsection\par
  \end{beamercolorbox}
}
\setbeamertemplate{subsection page}{
  \centering
  \begin{beamercolorbox}[sep=8pt,center]{subsection title}
    \usebeamerfont{subsection title}\insertsubsection\par
  \end{beamercolorbox}
}
% Prevent slide breaks in the middle of a paragraph
\widowpenalties 1 10000
\raggedbottom
\AtBeginPart{
  \frame{\partpage}
}
\AtBeginSection{
  \ifbibliography
  \else
    \frame{\sectionpage}
  \fi
}
\AtBeginSubsection{
  \frame{\subsectionpage}
}
\usepackage{iftex}
\ifPDFTeX
  \usepackage[T1]{fontenc}
  \usepackage[utf8]{inputenc}
  \usepackage{textcomp} % provide euro and other symbols
\else % if luatex or xetex
  \usepackage{unicode-math} % this also loads fontspec
  \defaultfontfeatures{Scale=MatchLowercase}
  \defaultfontfeatures[\rmfamily]{Ligatures=TeX,Scale=1}
\fi
\usepackage{lmodern}

\usetheme[]{Rochester}
\ifPDFTeX\else
  % xetex/luatex font selection
\fi
% Use upquote if available, for straight quotes in verbatim environments
\IfFileExists{upquote.sty}{\usepackage{upquote}}{}
\IfFileExists{microtype.sty}{% use microtype if available
  \usepackage[]{microtype}
  \UseMicrotypeSet[protrusion]{basicmath} % disable protrusion for tt fonts
}{}
\makeatletter
\@ifundefined{KOMAClassName}{% if non-KOMA class
  \IfFileExists{parskip.sty}{%
    \usepackage{parskip}
  }{% else
    \setlength{\parindent}{0pt}
    \setlength{\parskip}{6pt plus 2pt minus 1pt}}
}{% if KOMA class
  \KOMAoptions{parskip=half}}
\makeatother


\usepackage{longtable,booktabs,array}
\usepackage{calc} % for calculating minipage widths
\usepackage{caption}
% Make caption package work with longtable
\makeatletter
\def\fnum@table{\tablename~\thetable}
\makeatother
\usepackage{graphicx}
\makeatletter
\newsavebox\pandoc@box
\newcommand*\pandocbounded[1]{% scales image to fit in text height/width
  \sbox\pandoc@box{#1}%
  \Gscale@div\@tempa{\textheight}{\dimexpr\ht\pandoc@box+\dp\pandoc@box\relax}%
  \Gscale@div\@tempb{\linewidth}{\wd\pandoc@box}%
  \ifdim\@tempb\p@<\@tempa\p@\let\@tempa\@tempb\fi% select the smaller of both
  \ifdim\@tempa\p@<\p@\scalebox{\@tempa}{\usebox\pandoc@box}%
  \else\usebox{\pandoc@box}%
  \fi%
}
% Set default figure placement to htbp
\def\fps@figure{htbp}
\makeatother


% definitions for citeproc citations
\NewDocumentCommand\citeproctext{}{}
\NewDocumentCommand\citeproc{mm}{%
  \begingroup\def\citeproctext{#2}\cite{#1}\endgroup}
\makeatletter
 % allow citations to break across lines
 \let\@cite@ofmt\@firstofone
 % avoid brackets around text for \cite:
 \def\@biblabel#1{}
 \def\@cite#1#2{{#1\if@tempswa , #2\fi}}
\makeatother
\newlength{\cslhangindent}
\setlength{\cslhangindent}{1.5em}
\newlength{\csllabelwidth}
\setlength{\csllabelwidth}{3em}
\newenvironment{CSLReferences}[2] % #1 hanging-indent, #2 entry-spacing
 {\begin{list}{}{%
  \setlength{\itemindent}{0pt}
  \setlength{\leftmargin}{0pt}
  \setlength{\parsep}{0pt}
  % turn on hanging indent if param 1 is 1
  \ifodd #1
   \setlength{\leftmargin}{\cslhangindent}
   \setlength{\itemindent}{-1\cslhangindent}
  \fi
  % set entry spacing
  \setlength{\itemsep}{#2\baselineskip}}}
 {\end{list}}
\usepackage{calc}
\newcommand{\CSLBlock}[1]{\hfill\break\parbox[t]{\linewidth}{\strut\ignorespaces#1\strut}}
\newcommand{\CSLLeftMargin}[1]{\parbox[t]{\csllabelwidth}{\strut#1\strut}}
\newcommand{\CSLRightInline}[1]{\parbox[t]{\linewidth - \csllabelwidth}{\strut#1\strut}}
\newcommand{\CSLIndent}[1]{\hspace{\cslhangindent}#1}



\setlength{\emergencystretch}{3em} % prevent overfull lines

\providecommand{\tightlist}{%
  \setlength{\itemsep}{0pt}\setlength{\parskip}{0pt}}



 


\usepackage{xcolor}
\definecolor{mygreen}{RGB}{0,100,0}
\makeatletter
\@ifpackageloaded{caption}{}{\usepackage{caption}}
\AtBeginDocument{%
\ifdefined\contentsname
  \renewcommand*\contentsname{Table of contents}
\else
  \newcommand\contentsname{Table of contents}
\fi
\ifdefined\listfigurename
  \renewcommand*\listfigurename{List of Figures}
\else
  \newcommand\listfigurename{List of Figures}
\fi
\ifdefined\listtablename
  \renewcommand*\listtablename{List of Tables}
\else
  \newcommand\listtablename{List of Tables}
\fi
\ifdefined\figurename
  \renewcommand*\figurename{Figure}
\else
  \newcommand\figurename{Figure}
\fi
\ifdefined\tablename
  \renewcommand*\tablename{Table}
\else
  \newcommand\tablename{Table}
\fi
}
\@ifpackageloaded{float}{}{\usepackage{float}}
\floatstyle{ruled}
\@ifundefined{c@chapter}{\newfloat{codelisting}{h}{lop}}{\newfloat{codelisting}{h}{lop}[chapter]}
\floatname{codelisting}{Listing}
\newcommand*\listoflistings{\listof{codelisting}{List of Listings}}
\makeatother
\makeatletter
\makeatother
\makeatletter
\@ifpackageloaded{caption}{}{\usepackage{caption}}
\@ifpackageloaded{subcaption}{}{\usepackage{subcaption}}
\makeatother

\usepackage{bookmark}
\IfFileExists{xurl.sty}{\usepackage{xurl}}{} % add URL line breaks if available
\urlstyle{same}
\hypersetup{
  pdftitle={When Big Business Loses (?) : The Political Economy of Regulation in the Context of Noisy Politics},
  pdfauthor={Oscar Dumas; Hyung Won Kim; Nikolaos Vichos},
  hidelinks,
  pdfcreator={LaTeX via pandoc}}


\title{When Big Business Loses (?) : The Political Economy of Regulation
in the Context of Noisy Politics}
\author{Oscar Dumas \and Hyung Won Kim \and Nikolaos Vichos}
\date{}

\begin{document}
\frame{\titlepage}


\begin{frame}[fragile]{Overview of Presentation}
\phantomsection\label{overview-of-presentation}
\begin{enumerate}
\tightlist
\item
  Introduction
\item
  Theoretical Framework and Important Concepts: Noisy Politics
\item
  Side-By-Side: Paper Summaries and Critical Discussion
\item
  Synthesis: What Unites the Two Papers?
\item
  Contextualization: Noisy Politics and the Political Economy of
  Regulation
\item
  Conclusion and Discussion
\end{enumerate}
\end{frame}

\section{Introduction}\label{introduction}

\begin{frame}{Texts}
\phantomsection\label{texts}
\begin{itemize}
\tightlist
\item
  Massoc, Elsa. 2019. ``Taxing Stock Transfers in the First Golden Age
  of Financial Capitalism: Political Salience and the Limits on the
  Power of Finance.'' \emph{Socio-Economic Review} 17 (3): 503--22.
  https://doi.org/10.1093/ser/mwx039.
\item
  Feldmann, Magnus, and Glenn Morgan. 2021. ``Brexit and British
  Business Elites: Business Power and Noisy Politics.'' \emph{Politics
  \& Society} 49 (1): 107--31. https://doi.org/10.1177/0032329220985692.
\end{itemize}
\end{frame}

\begin{frame}{Research Focus}
\phantomsection\label{research-focus}
\begin{block}{Massoc (2019) :}
\phantomsection\label{massoc2019}
\begin{itemize}
\tightlist
\item
  Why did powerful financial actors fail to prevent the adoption of
  stock-transfer taxes in France and New York?
\item
  \textbf{Focus}: how public discontent + political framing raised
  salience and undermined ``quiet politics.''
\end{itemize}
\end{block}

\begin{block}{Feldmann and Morgan (2021):}
\phantomsection\label{feldmann2021}
\begin{itemize}
\tightlist
\item
  Under what conditions can business elites be effective in a context of
  noisy politics, and why did their influence differ between 1975 and
  2016?
\item
  \textbf{Focus}: how incentives, legitimacy, and cohesion shape
  business influence in high-politicization referendum campaigns.
\end{itemize}
\end{block}
\end{frame}

\begin{frame}{What brings these papers together?}
\phantomsection\label{what-brings-these-papers-together}
Both papers examine how political visibility and public contestation
reshape business/financial influence

\begin{itemize}
\item
  Massoc (2019) shows how politicization of a tax issue enabled
  politicians to override strong financial lobbying.
\item
  Feldmann and Morgan (2021) analyze how changing political conditions
  over time affect business effectiveness in high-visibility referendum
  campaigns.\\
\end{itemize}

\begin{block}{\textbf{Common Question}}
\phantomsection\label{common-question}
\begin{itemize}
\tightlist
\item
  \emph{Under what conditions do business actors struggle to assert
  influence?}
\end{itemize}
\end{block}
\end{frame}

\section{Theoretical Framework and Important Concepts: Noisy
Politics}\label{theoretical-framework-and-important-concepts-noisy-politics}

\begin{frame}{Culpepper's Framework: Quiet vs.~Noisy Politics}
\phantomsection\label{culpeppers-framework-quiet-vs.-noisy-politics}
\textbf{Core hypothesis:} The more the public cares about an issue, the
less business can exercise influence over regulation. Business power
decreases as political salience increases (Culpepper 2021). \small

\begin{tabular}{p{0.48\textwidth} p{0.48\textwidth}}
\hline
\textbf{Quiet Politics} & \textbf{Noisy Politics} \\
\hline
Issues are technically complex and low in public salience. Business dominates through expertise monopoly and structural dependence. &
Issues become publicly salient and politicized. Media attention and public engagement constrain business influence and organizations become cautious and may avoid public positioning. \\
\hline
\end{tabular}

\normalsize

\begin{itemize}
\tightlist
\item
  \textbf{Three forms of business power}: structural, instrumental, and
  ideational
\item
  \textbf{The challenge}: Business power persists even in noisy politics
  through technical complexity, strategic network-building, and control
  of core growth sectors.
\end{itemize}
\end{frame}

\section{Side-By-Side: Paper Summaries and Critical
Discussio}\label{side-by-side-paper-summaries-and-critical-discussio}

\begin{frame}{Feldmann and Morgan (2021) : Brexit and British Business
Elites}
\phantomsection\label{feldmann2021-brexit-and-british-business-elites}
\textbf{The question}: When can business elites influence ``noisy
politics''? Comparing EU Referenda

\textbf{Three determinants of business effectiveness}:

\begin{center}
\includegraphics[width=3.4375in,height=\textheight,keepaspectratio]{Screenshot 2025-11-18 at 14.13.03.png}
\end{center}

\textbf{The mechanism}

\begin{itemize}
\item
  1975: Unified business presence in workplaces → counter-framing
  succeeds
\item
  2016: Fragmented business silent or hedging → anti-establishment
  framing dominates
\end{itemize}
\end{frame}

\begin{frame}{Massoc (2019) \textbf{:} Stock Transfer Tax \& Political
Salience}
\phantomsection\label{massoc2019-stock-transfer-tax-political-salience}
\textbf{The puzzle:} Same anti-finance sentiment → Different outcomes.

\textbf{Why?}

\begin{block}{1903: Stock transfer tax proposed}
\phantomsection\label{stock-transfer-tax-proposed}
↓ Public anger:
\textcolor{mygreen}{Financial crisis + anti-finance sentiment}

↓ Finance opposition: \textcolor{mygreen}{Massive}

→ \textbf{Outcome:} \textcolor{red}{Rejected}
\end{block}

\begin{block}{1905: Stock transfer tax proposed (same tax!)}
\phantomsection\label{stock-transfer-tax-proposed-same-tax}
↓ Public anger: \textcolor{mygreen}{Same level}

↓ Finance opposition: \textcolor{mygreen}{Same level}

→Difference: Political champions (Senators Lewis \& Raines) publicly
framed issue

→ \textbf{Outcome:} \textcolor{mygreen}{Passed}
\end{block}
\end{frame}

\begin{frame}{The answer: Framing creates salience}
\phantomsection\label{the-answer-framing-creates-salience}
It requires political entrepreneurs who:

\begin{itemize}
\item
  Reframe complexity as simple antagonism (``finance vs.~people'')
\item
  Have incentive to alienate finance (party positioning)
\item
  Actively publicize the issue (top-down process)
\item
  Same structural conditions; different political choices = different
  outcomes
\end{itemize}
\end{frame}

\begin{frame}{Critical review -- What actually causes business defeat?}
\phantomsection\label{critical-review-what-actually-causes-business-defeat}
\textbf{Framing as a cause -- Massoc's logic:}

\begin{itemize}
\item
  \textcolor{mygreen}{Shows political agency matters}
\item
  \textcolor{mygreen}{Same conditions → different framing = different outcomes}
\item
  \textcolor{red}{Does not explain why framing succeeds in some contexts not others}
\end{itemize}

\textbf{Elite fragmentation as a cause -- Feldmann \& Morgan's logic:}

\begin{itemize}
\item
  \textcolor{mygreen}{Shows structural vulnerability enables defeats}
\item
  \textcolor{mygreen}{975 cohesion defeats framing attempts; 2016 fragmentation allows framing to succeed}
\item
  \textcolor{red}{Does not explain how fragmentation becomes politically visible}
\end{itemize}
\end{frame}

\section{Synthesis: What Brings These Papers
Together?}\label{synthesis-what-brings-these-papers-together}

\begin{frame}{Two Sides of the Business--Politics Struggle}
\phantomsection\label{two-sides-of-the-businesspolitics-struggle}
The papers map the conditions that shift the balance in favor of
politicians or business.~

\begin{block}{\textbf{Two axes of analysis}}
\phantomsection\label{two-axes-of-analysis}
\begin{itemize}
\item
  \emph{``Anti-Business Offense'' vs ``Business Defense''}

  \begin{itemize}
  \item
    Massoc: How can political actors mount an effective anti-business
    offense?
  \item
    Feldmann and Morgan: When can business can (or cannot) successfully
    defend itself in noisy politics?~~
  \end{itemize}
\item
  \emph{Noisy Politics Manicheanism: From New York and France to Brexit}
\end{itemize}
\end{block}
\end{frame}

\begin{frame}{Anti-Business Offense'' vs ``Business Defense''}
\phantomsection\label{anti-business-offense-vs-business-defense}
\textbf{Massoc: Anti-Business Offense}

\begin{itemize}
\item
  Offense is politically constructed, not automatic.
\item
  Politicians create salience and frame issues in moral, binary terms.
\item
  Framing channels diffuse public anger into a focused attack on
  finance.
\end{itemize}

\textbf{Feldmann and Morgan: Business Defense}

Business defense works only when three conditions align:

\begin{enumerate}
\item
  Incentives to engage --- business must see existential stakes.
\item
  Legitimacy --- messages must be perceived as serving the public, not
  narrow interests.
\item
  Cohesion --- unified signaling increases credibility and influence
\end{enumerate}
\end{frame}

\begin{frame}{Anti-Business Offense'' vs ``Business Defense''}
\phantomsection\label{anti-business-offense-vs-business-defense-1}
\begin{itemize}
\item
  Massoc (2019) explains how politicians win by activating the offense
  whereas Feldmann and Morgan (2021) explain how business can (or
  cannot) counterattack.
\item
  Offense gains power when issues are moralized, simplified, and made
  public. Under the same conditions, business loses:

  \begin{itemize}
  \item
    Harder to mobilize
  \item
    Less legitimate
  \item
    More divided
  \end{itemize}
\end{itemize}

\begin{block}{In brief}
\phantomsection\label{in-brief}
\begin{itemize}
\item
  Massoc describes the conditions for business defeat; Feldmann and
  Morgan describe the mechanics of business resistance.
\item
  A full political-economy picture emerges only when offense and defense
  are examined together.
\end{itemize}
\end{block}
\end{frame}

\begin{frame}{Manicheanism}
\phantomsection\label{manicheanism}
\begin{longtable}[]{@{}
  >{\raggedright\arraybackslash}p{(\linewidth - 0\tabcolsep) * \real{1.0060}}@{}}
\toprule\noalign{}
\endhead
\emph{``Politics is a struggle between good and evil. {[}\ldots{]} The
Manichean worldview sees the struggle between opposing camps as an
inherently moral one.''} (Jungkunz, Fahey, and Hino 2021) \\
\bottomrule\noalign{}
\end{longtable}
\end{frame}

\begin{frame}{Manicheanism in Massoc (2019) and Feldmann and Morgan
(2021)}
\phantomsection\label{manicheanism-in-massoc2019-and-feldmann2021}
\begin{itemize}
\item
  Massoc (2019)

  \begin{itemize}
  \item
    Central to the offense: ``Speculators vs.~workers''
  \item
    Turns a technical tax into a moral crusade, neutralizing expert
    arguments.
  \end{itemize}
\item
  Feldmann and Morgan (2021)

  \begin{itemize}
  \item
    Not named as ``manicheanism,'' but structurally identical: Brexit as
    ``the people vs.~the establishment.''
  \item
    Business expertise becomes delegitimized (``Project Fear'').
  \item
    Manicheanism collapses the three conditions for effective business
    defense.
  \end{itemize}
\end{itemize}
\end{frame}

\begin{frame}{Synthesizing Manicheanism}
\phantomsection\label{synthesizing-manicheanism}
\textbf{Manicheanism as the bridge between defense and offense}

\begin{itemize}
\item
  Manichean framing empowers anti-business offense
\item
  It simultaneously undermines business defense by:

  \begin{itemize}
  \item
    Making business appear self-interested
  \item
    Casting warnings as elite manipulation
  \item
    Punishing technocratic/complex messaging
  \item
    Fragmenting business coalitions
  \end{itemize}
\end{itemize}

\textbf{Across contexts separated by a century, Manichean politics
reorders power:}

\begin{itemize}
\item
  Politicians gain room to attack
\item
  Business loses the capacity to respond
\end{itemize}
\end{frame}

\section{Contextualization: Noisy Politics and the Political Economy of
Regulation}\label{contextualization-noisy-politics-and-the-political-economy-of-regulation}

\begin{frame}{The Shift of the ``Big Gun'': From Legal Sanctions
(Session 5) to Political Strikes (Session 9)}
\phantomsection\label{the-shift-of-the-big-gun-from-legal-sanctions-session-5-to-political-strikes-session-9}
\tiny

\begin{longtable}[]{@{}
  >{\raggedright\arraybackslash}p{(\linewidth - 6\tabcolsep) * \real{0.1436}}
  >{\raggedright\arraybackslash}p{(\linewidth - 6\tabcolsep) * \real{0.2873}}
  >{\raggedright\arraybackslash}p{(\linewidth - 6\tabcolsep) * \real{0.2652}}
  >{\raggedright\arraybackslash}p{(\linewidth - 6\tabcolsep) * \real{0.2928}}@{}}
\toprule\noalign{}
\begin{minipage}[b]{\linewidth}\raggedright
\textbf{Category}
\end{minipage} & \begin{minipage}[b]{\linewidth}\raggedright
\textbf{Legal Sanctions:}

\textbf{Institutional Ideals}
\end{minipage} & \begin{minipage}[b]{\linewidth}\raggedright
\textbf{Reality:}

\textbf{The Breakdown}
\end{minipage} & \begin{minipage}[b]{\linewidth}\raggedright
\textbf{Political Strikes:}

\textbf{The Shift to Noise}
\end{minipage} \\
\midrule\noalign{}
\endhead
\textbf{Regulatory Model} & \textbf{Benign Big Gun} (Ayres et al. 1995)

Strong legal sanctions such as license revocation & \textbf{Broken Gun}

Legal power disabled due to `Too Big to Fail' &
\begin{minipage}[t]{\linewidth}\raggedright
\textbf{Political Big Gun} (Massoc 2019; Feldmann and Morgan 2021)

Political strike using Public Salience\\
\strut
\end{minipage} \\
\begin{minipage}[t]{\linewidth}\raggedright
\textbf{Regulatory Actor\\
}\strut
\end{minipage} & \textbf{Organized PIGs}

(Public Interest Groups) & - & \textbf{Unorganized Public} \\
\begin{minipage}[t]{\linewidth}\raggedright
\textbf{Regulatory Mechanism\\
}\strut
\end{minipage} & \begin{minipage}[t]{\linewidth}\raggedright
\textbf{Institutional deterrence}

Constant surveillance within the pyramid\\
\strut
\end{minipage} & \textbf{Loss of regulator's bargaining power} &
\textbf{Non-institutional Explosion}

External pressure via public mobilization) \\
\begin{minipage}[t]{\linewidth}\raggedright
\textbf{Target Behavior\\
}\strut
\end{minipage} & \textbf{Voluntary Compliance} & \textbf{Evasion or
Capture} & \textbf{Coerced Submission} \\
\textbf{Regulatory Effect} & \textbf{Stable Deterrence}

\textbf{+ High sustainability} & \textbf{Regulatory Void} &
\textbf{Temporary impact} (vulnerable to rollback)

\textbf{+ High volatility} \\
\bottomrule\noalign{}
\end{longtable}

\normalsize
\end{frame}

\begin{frame}[allowframebreaks]{Bibliography}
\phantomsection\label{bibliography}
\phantomsection\label{refs}
\begin{CSLReferences}{1}{0}
\bibitem[\citeproctext]{ref-ayres1995}
Ayres, Ian, John Braithwaite, Ian Ayres, and John Braithwaite. 1995.
\emph{Responsive Regulation: Transcending the Deregulation Debate}.
Oxford Socio-Legal Studies. Oxford, New York: Oxford University Press.

\bibitem[\citeproctext]{ref-culpepper2021}
Culpepper, Pepper D. 2021. {``Quiet Politics in Tumultuous Times:
Business Power, Populism, and Democracy*.''} \emph{Politics \& Society}
49 (1): 133--43. \url{https://doi.org/10.1177/0032329220985725}.

\bibitem[\citeproctext]{ref-feldmann2021}
Feldmann, Magnus, and Glenn Morgan. 2021. {``Brexit and British Business
Elites: Business Power and Noisy Politics*.''} \emph{Politics \&
Society} 49 (1): 107--31.
\url{https://doi.org/10.1177/0032329220985692}.

\bibitem[\citeproctext]{ref-jungkunz2021}
Jungkunz, Sebastian, Robert A. Fahey, and Airo Hino. 2021. {``How
Populist Attitudes Scales Fail to Capture Support for Populists in
Power.''} \emph{PLoS ONE} 16 (12): e0261658.
\url{https://doi.org/10.1371/journal.pone.0261658}.

\bibitem[\citeproctext]{ref-massoc2019}
Massoc, Elsa. 2019. {``Taxing Stock Transfers in the First Golden Age of
Financial Capitalism: Political Salience and the Limits on the Power of
Finance.''} \emph{Socio-Economic Review} 17 (3): 503--22.
\url{https://doi.org/10.1093/ser/mwx039}.

\end{CSLReferences}
\end{frame}




\end{document}
