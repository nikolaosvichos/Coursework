% Options for packages loaded elsewhere
% Options for packages loaded elsewhere
\PassOptionsToPackage{unicode}{hyperref}
\PassOptionsToPackage{hyphens}{url}
%
\documentclass[
  ignorenonframetext,
]{beamer}
\newif\ifbibliography
\usepackage{pgfpages}
\setbeamertemplate{caption}[numbered]
\setbeamertemplate{caption label separator}{: }
\setbeamercolor{caption name}{fg=normal text.fg}
\beamertemplatenavigationsymbolsempty
% remove section numbering
\setbeamertemplate{part page}{
  \centering
  \begin{beamercolorbox}[sep=16pt,center]{part title}
    \usebeamerfont{part title}\insertpart\par
  \end{beamercolorbox}
}
\setbeamertemplate{section page}{
  \centering
  \begin{beamercolorbox}[sep=12pt,center]{section title}
    \usebeamerfont{section title}\insertsection\par
  \end{beamercolorbox}
}
\setbeamertemplate{subsection page}{
  \centering
  \begin{beamercolorbox}[sep=8pt,center]{subsection title}
    \usebeamerfont{subsection title}\insertsubsection\par
  \end{beamercolorbox}
}
% Prevent slide breaks in the middle of a paragraph
\widowpenalties 1 10000
\raggedbottom
\AtBeginPart{
  \frame{\partpage}
}
\AtBeginSection{
  \ifbibliography
  \else
    \frame{\sectionpage}
  \fi
}
\AtBeginSubsection{
  \frame{\subsectionpage}
}
\usepackage{iftex}
\ifPDFTeX
  \usepackage[T1]{fontenc}
  \usepackage[utf8]{inputenc}
  \usepackage{textcomp} % provide euro and other symbols
\else % if luatex or xetex
  \usepackage{unicode-math} % this also loads fontspec
  \defaultfontfeatures{Scale=MatchLowercase}
  \defaultfontfeatures[\rmfamily]{Ligatures=TeX,Scale=1}
\fi
\usepackage{lmodern}

\usetheme[]{Rochester}
\ifPDFTeX\else
  % xetex/luatex font selection
\fi
% Use upquote if available, for straight quotes in verbatim environments
\IfFileExists{upquote.sty}{\usepackage{upquote}}{}
\IfFileExists{microtype.sty}{% use microtype if available
  \usepackage[]{microtype}
  \UseMicrotypeSet[protrusion]{basicmath} % disable protrusion for tt fonts
}{}
\makeatletter
\@ifundefined{KOMAClassName}{% if non-KOMA class
  \IfFileExists{parskip.sty}{%
    \usepackage{parskip}
  }{% else
    \setlength{\parindent}{0pt}
    \setlength{\parskip}{6pt plus 2pt minus 1pt}}
}{% if KOMA class
  \KOMAoptions{parskip=half}}
\makeatother


\usepackage{longtable,booktabs,array}
\usepackage{calc} % for calculating minipage widths
\usepackage{caption}
% Make caption package work with longtable
\makeatletter
\def\fnum@table{\tablename~\thetable}
\makeatother
\usepackage{graphicx}
\makeatletter
\newsavebox\pandoc@box
\newcommand*\pandocbounded[1]{% scales image to fit in text height/width
  \sbox\pandoc@box{#1}%
  \Gscale@div\@tempa{\textheight}{\dimexpr\ht\pandoc@box+\dp\pandoc@box\relax}%
  \Gscale@div\@tempb{\linewidth}{\wd\pandoc@box}%
  \ifdim\@tempb\p@<\@tempa\p@\let\@tempa\@tempb\fi% select the smaller of both
  \ifdim\@tempa\p@<\p@\scalebox{\@tempa}{\usebox\pandoc@box}%
  \else\usebox{\pandoc@box}%
  \fi%
}
% Set default figure placement to htbp
\def\fps@figure{htbp}
\makeatother


% definitions for citeproc citations
\NewDocumentCommand\citeproctext{}{}
\NewDocumentCommand\citeproc{mm}{%
  \begingroup\def\citeproctext{#2}\cite{#1}\endgroup}
\makeatletter
 % allow citations to break across lines
 \let\@cite@ofmt\@firstofone
 % avoid brackets around text for \cite:
 \def\@biblabel#1{}
 \def\@cite#1#2{{#1\if@tempswa , #2\fi}}
\makeatother
\newlength{\cslhangindent}
\setlength{\cslhangindent}{1.5em}
\newlength{\csllabelwidth}
\setlength{\csllabelwidth}{3em}
\newenvironment{CSLReferences}[2] % #1 hanging-indent, #2 entry-spacing
 {\begin{list}{}{%
  \setlength{\itemindent}{0pt}
  \setlength{\leftmargin}{0pt}
  \setlength{\parsep}{0pt}
  % turn on hanging indent if param 1 is 1
  \ifodd #1
   \setlength{\leftmargin}{\cslhangindent}
   \setlength{\itemindent}{-1\cslhangindent}
  \fi
  % set entry spacing
  \setlength{\itemsep}{#2\baselineskip}}}
 {\end{list}}
\usepackage{calc}
\newcommand{\CSLBlock}[1]{\hfill\break\parbox[t]{\linewidth}{\strut\ignorespaces#1\strut}}
\newcommand{\CSLLeftMargin}[1]{\parbox[t]{\csllabelwidth}{\strut#1\strut}}
\newcommand{\CSLRightInline}[1]{\parbox[t]{\linewidth - \csllabelwidth}{\strut#1\strut}}
\newcommand{\CSLIndent}[1]{\hspace{\cslhangindent}#1}



\setlength{\emergencystretch}{3em} % prevent overfull lines

\providecommand{\tightlist}{%
  \setlength{\itemsep}{0pt}\setlength{\parskip}{0pt}}



 


\makeatletter
\@ifpackageloaded{caption}{}{\usepackage{caption}}
\AtBeginDocument{%
\ifdefined\contentsname
  \renewcommand*\contentsname{Table of contents}
\else
  \newcommand\contentsname{Table of contents}
\fi
\ifdefined\listfigurename
  \renewcommand*\listfigurename{List of Figures}
\else
  \newcommand\listfigurename{List of Figures}
\fi
\ifdefined\listtablename
  \renewcommand*\listtablename{List of Tables}
\else
  \newcommand\listtablename{List of Tables}
\fi
\ifdefined\figurename
  \renewcommand*\figurename{Figure}
\else
  \newcommand\figurename{Figure}
\fi
\ifdefined\tablename
  \renewcommand*\tablename{Table}
\else
  \newcommand\tablename{Table}
\fi
}
\@ifpackageloaded{float}{}{\usepackage{float}}
\floatstyle{ruled}
\@ifundefined{c@chapter}{\newfloat{codelisting}{h}{lop}}{\newfloat{codelisting}{h}{lop}[chapter]}
\floatname{codelisting}{Listing}
\newcommand*\listoflistings{\listof{codelisting}{List of Listings}}
\makeatother
\makeatletter
\makeatother
\makeatletter
\@ifpackageloaded{caption}{}{\usepackage{caption}}
\@ifpackageloaded{subcaption}{}{\usepackage{subcaption}}
\makeatother

\usepackage{bookmark}
\IfFileExists{xurl.sty}{\usepackage{xurl}}{} % add URL line breaks if available
\urlstyle{same}
\hypersetup{
  pdftitle={Coming of Age Under Trump},
  pdfauthor={Nikolaos Vichos},
  hidelinks,
  pdfcreator={LaTeX via pandoc}}


\title{Coming of Age Under Trump}
\subtitle{The Effect of First Electoral Exposure in a Trump Election on
Future Media Trust}
\author{Nikolaos Vichos}
\date{November 21, 2025}

\begin{document}
\frame{\titlepage}


\begin{frame}{Overview}
\phantomsection\label{overview}
\begin{enumerate}
\tightlist
\item
  Context
\item
  Literature Review
\item
  Theoretical Framework
\item
  Research Design
\end{enumerate}
\end{frame}

\section{Context}\label{context}

\begin{frame}{2016 U.S. Presidential Election}
\phantomsection\label{u.s.-presidential-election}
\begin{itemize}
\item
  The 2016 election was an unprecedented rupture, marked by extreme
  polarization and norm violations.
\item
  Trump's campaign used racialized appeals, anti-immigration rhetoric,
  and, most importantly, attacks on instutions and the media.
\item
  Long-standing political norms---transparency, restraint,
  constitutional conventions---were openly disregarded.
\item
  Young voters entering political life in 2016 faced a uniquely
  antagonistic and norm-breaking environment.
\end{itemize}

\begin{block}{Research question}
\phantomsection\label{research-question}
\emph{Does coming of age during this disruptive election affect
long-term media trust?}
\end{block}
\end{frame}

\section{Literature Review}\label{literature-review}

\begin{frame}{Literature Review (Formative Periods \& Political
Development)}
\phantomsection\label{literature-review-formative-periods-political-development}
\small

\textbf{Early Life \& Early Adulthood}

\begin{itemize}
\tightlist
\item
  Early adversity shapes long-term attitudes/behaviors (Shonkoff et al.
  2012; Bellucci, Fuochi, and Conzo 2020)
\item
  Early adulthood = second sensitive period; identities remain malleable
  (Masten et al. 2004; Arnett 2000)
\end{itemize}

\textbf{Political Development}

\begin{itemize}
\tightlist
\item
  First political encounters crystallize worldviews (Mannheim 1928)
\item
  Civic experiences and social environment shape long-term engagement
  (Wray-Lake and Ballard 2023; Sagawa 2010)
\item
  Cohort effects: political climate of youth has lasting influence
  (Ghitza, Gelman, and Auerbach 2023; Grasso 2014)
\end{itemize}

\textbf{First Election Effects}

\begin{itemize}
\tightlist
\item
  First elections form lasting turnout habits and shape partisanship
  (Plutzer 2002; Dinas 2012; Meredith 2009)
\end{itemize}

\normalsize
\end{frame}

\begin{frame}{Literature Review (Elite Norms \& Democratic Backsliding)}
\phantomsection\label{literature-review-elite-norms-democratic-backsliding}
\textbf{Elite Cues \& Public Response}

\begin{itemize}
\tightlist
\item
  Citizens respond to elite norm violations, legitimizing or resisting
  them (Druckman 2024)
\item
  Fear of opposing party subversion increases tolerance for undemocratic
  actions (Braley et al. 2023)
\item
  Polarization reduces willingness to punish copartisans for violations
  (Graham and Svolik 2020)
\end{itemize}

\textbf{Relevance to 2016}

\begin{itemize}
\tightlist
\item
  Trump era = major elite norm breach.
\item
  Exposure during coming-of-age may shape long-term support for media
  freedom.
\item
  Study examines whether ages 18--25 in this context altered media
  attitudes.
\end{itemize}
\end{frame}

\section{Theoretical Framework}\label{theoretical-framework}

\begin{frame}{Theory and Hypotheses}
\phantomsection\label{theory-and-hypotheses}
\small

\textbf{First Encounters with Democracy}

\begin{itemize}
\tightlist
\item
  \textbf{H1:} Negative effect on media attitudes \emph{(`first-time
  sensitivity')}.
\end{itemize}

\textbf{Partisan Sensitivity}

\begin{itemize}
\item
  Partisans more responsive to political cues; independents less engaged
  and influenced (Magleby, Nelson, and Westlye 2011; Mandel and Omorogbe
  2014)
\item
  \textbf{H2:} First-time voting effect stronger for partisans than
  independents \emph{(`non-partisan insensitivity').}
\end{itemize}

\textbf{Asymmetric Sensitivity}

\begin{itemize}
\tightlist
\item
  Republicans more affectively sensitive to political context than
  Democrats (Carraro, Castelli, and Macchiella 2011; Mandel and Omorogbe
  2014) (and more to Trump cues)
\item
  \textbf{H3:} Within partisans, effect stronger for Republicans than
  Democrats (\emph{`asymmetric sensitivity').}
\end{itemize}

\normalsize
\end{frame}

\section{Research Design}\label{research-design}

\begin{frame}{Methodology and Data}
\phantomsection\label{methodology-and-data}
\textbf{Data}

\begin{itemize}
\tightlist
\item
  2020 ANES
\end{itemize}

\textbf{Outcome: Media Trust Index}

\begin{itemize}
\item
  Constructed from 2020 ANES items on media trust, journalist access,
  and concerns about media freedom.
\item
  0--1 continuous index (factor analysis)
\end{itemize}

\textbf{Design}

\begin{itemize}
\item
  Sharp RDD compares those eligible to vote in 2012 vs.~2016 (age 26
  cutoff in 2020).
\item
  Simulates as-if random assignment to ``conventional''
  vs.~``unconventional'' political environment
\end{itemize}
\end{frame}

\begin{frame}{RDD Equation}
\phantomsection\label{rdd-equation}
\[
Y_i = \alpha + \tau D_i + f(X_i - c) + \varepsilon_i
\]

\small

Where:

\begin{itemize}
\tightlist
\item
  \(Y_i\) is the outcome, a continuous index measuring liberal
  attitudes.
\item
  \(D_i\) is an indicator equal to 1 if individual \(i\)'s age
  \emph{exceeds} the cutoff (i.e., they were just \emph{eligible} to
  vote in 2012), and 0 otherwise.{[}\^{}3{]}
\item
  \(\tau\) is the parameter of interest, capturing the size of the
  discontinuity at the threshold---that is, the causal effect of being
  first eligible to vote in 2012 rather than in 2016.
\item
  \(X_i\) is the running variable, the respondent's age in years in
  2020.
\item
  \(f(X_i - c)\) denotes a flexible function (e.g., first- or
  second-order polynomial) of the centered running variable---distance
  from the cutoff \(c\).
\item
  \(c = 26\) is the cutoff point, so \(X_i - c\) represents the running
  variable centered at the threshold.
\item
  \(\varepsilon_i\) is an error term.
\end{itemize}

\normalsize
\end{frame}

\section{Preliminary Results}\label{preliminary-results}

\begin{frame}{Results for Similar Index (Liberal Democracy)}
\phantomsection\label{results-for-similar-index-liberal-democracy}
\begin{center}
\includegraphics[width=4.16667in,height=\textheight,keepaspectratio]{Prelim_results_liberalism.png}
\end{center}
\end{frame}

\begin{frame}{Questions/Concerns}
\phantomsection\label{questionsconcerns}
\begin{itemize}
\tightlist
\item
  Sharp RDD according to eligibility or Fuzzy RDD according to
  first-time voting?
\end{itemize}
\end{frame}

\begin{frame}[allowframebreaks]{Works Cited}
\phantomsection\label{works-cited}
\phantomsection\label{refs}
\begin{CSLReferences}{1}{0}
\bibitem[\citeproctext]{ref-arnett2000}
Arnett, Jeffrey Jensen. 2000. {``Emerging Adulthood: A Theory of
Development from the Late Teens Through the Twenties.''} \emph{American
Psychologist} 55 (5): 469--80.
\url{https://doi.org/10.1037/0003-066X.55.5.469}.

\bibitem[\citeproctext]{ref-bellucci2020}
Bellucci, Davide, Giulia Fuochi, and Pierluigi Conzo. 2020. {``Childhood
Exposure to the Second World War and Financial Risk Taking in Adult
Life.''} \emph{Journal of Economic Psychology}, The impact of life
experiences on risk taking, 79 (August): 102196.
\url{https://doi.org/10.1016/j.joep.2019.102196}.

\bibitem[\citeproctext]{ref-braley2023}
Braley, Alia, Gabriel S. Lenz, Dhaval Adjodah, Hossein Rahnama, and Alex
Pentland. 2023. {``Why voters who value democracy participate in
democratic backsliding.''} \emph{Nature Human Behaviour} 7 (8):
1282--93. \url{https://doi.org/10.1038/s41562-023-01594-w}.

\bibitem[\citeproctext]{ref-carraro2011}
Carraro, Luciana, Luigi Castelli, and Claudia Macchiella. 2011. {``The
Automatic Conservative: Ideology-Based Attentional Asymmetries in the
Processing of Valenced Information.''} \emph{PLOS ONE} 6 (11): e26456.
\url{https://doi.org/10.1371/journal.pone.0026456}.

\bibitem[\citeproctext]{ref-dinas2012}
Dinas, Elias. 2012. {``The Formation of Voting Habits.''} \emph{Journal
of Elections, Public Opinion and Parties} 22 (4): 431--56.
\url{https://doi.org/10.1080/17457289.2012.718280}.

\bibitem[\citeproctext]{ref-druckman2024}
Druckman, James N. 2024. {``How to Study Democratic Backsliding.''}
\emph{Political Psychology} 45 (S1): 3--42.
\url{https://doi.org/10.1111/pops.12942}.

\bibitem[\citeproctext]{ref-ghitza2023}
Ghitza, Yair, Andrew Gelman, and Jonathan Auerbach. 2023. {``The Great
Society, Reagan's Revolution, and Generations of Presidential Voting.''}
\emph{American Journal of Political Science} 67 (3): 520--37.
\url{https://doi.org/10.1111/ajps.12713}.

\bibitem[\citeproctext]{ref-graham2020}
Graham, Matthew H., and Milan W. Svolik. 2020. {``Democracy in America?
Partisanship, Polarization, and the Robustness of Support for Democracy
in the United States.''} \emph{American Political Science Review} 114
(2): 392--409. \url{https://doi.org/10.1017/S0003055420000052}.

\bibitem[\citeproctext]{ref-grasso2014}
Grasso, Maria T. 2014. {``Age, Period and Cohort Analysis in a
Comparative Context: Political Generations and Political Participation
Repertoires in Western Europe.''} \emph{Electoral Studies} 33 (March):
63--76. \url{https://doi.org/10.1016/j.electstud.2013.06.003}.

\bibitem[\citeproctext]{ref-magleby2011}
Magleby, David B., Candice J. Nelson, and Mark C. Westlye. 2011. {``The
Myth of the Independent Voter Revisited.''} In, edited by Paul M.
Sniderman and Benjamin Highton, 0. Princeton University Press.
\url{https://doi.org/10.23943/princeton/9780691151106.003.0011}.

\bibitem[\citeproctext]{ref-mandel2014}
Mandel, David R., and Philip Omorogbe. 2014. {``Political Differences in
Past, Present, and Future Life Satisfaction: Republicans Are More
Sensitive Than Democrats to Political Climate.''} \emph{PLOS ONE} 9 (6):
e98854. \url{https://doi.org/10.1371/journal.pone.0098854}.

\bibitem[\citeproctext]{ref-mannheim1928}
Mannheim, Karl. 1928. \emph{Le problème des générations}. Paris: Armand
Colin.

\bibitem[\citeproctext]{ref-masten2004}
Masten, Ann S., Keith B. Burt, Glenn I. Roisman, Jelena Obradović,
Jeffrey D. Long, and Auke Tellegen. 2004. {``Resources and resilience in
the transition to adulthood: continuity and change.''} \emph{Development
and Psychopathology} 16 (4): 1071--94.
\url{https://doi.org/10.1017/s0954579404040143}.

\bibitem[\citeproctext]{ref-meredith2009}
Meredith, Marc. 2009. {``Persistence in Political Participation.''}
\emph{Quarterly Journal of Political Science} 4 (3): 187--209.
\url{https://doi.org/10.1561/100.00009015}.

\bibitem[\citeproctext]{ref-plutzer2002}
Plutzer, Eric. 2002. {``Becoming a Habitual Voter: Inertia, Resources,
and Growth in Young Adulthood.''} \emph{The American Political Science
Review} 96 (1): 41--56. \url{https://www.jstor.org/stable/3117809}.

\bibitem[\citeproctext]{ref-sagawa2010}
Sagawa, Shirley. 2010. \emph{The American Way to Change: How National
Service and Volunteers Are Transforming America}. John Wiley; Sons.

\bibitem[\citeproctext]{ref-shonkoff2012}
Shonkoff, Jack P., Andrew S. Garner, Committee on Psychosocial Aspects
of Child, Family Health, Committee on Early Childhood, Adoption, and
Dependent Care, Section on Developmental, and Behavioral Pediatrics.
2012. {``The lifelong effects of early childhood adversity and toxic
stress.''} \emph{Pediatrics} 129 (1): e232--246.
\url{https://doi.org/10.1542/peds.2011-2663}.

\bibitem[\citeproctext]{ref-wray-lake2023}
Wray-Lake, Laura, and Parissa J. Ballard. 2023. {``Civic Engagement
Across Adolescence and Early Adulthood.''} In, edited by Lisa J.
Crockett, Gustavo Carlo, and John E. Schulenberg, 573--93. Washington:
American Psychological Association.
\url{https://doi.org/10.1037/0000298-035}.

\end{CSLReferences}
\end{frame}




\end{document}
